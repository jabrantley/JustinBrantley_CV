
% KORDING LAB
\researchexperience{Postdoctoral Researcher}{K\"{o}rding Lab \href{http://kordinglab.com/}{\linkicon}, Department of Bioengineering, University of Pennsylvania}{Feb 2020--Present}{Supervisor: Konrad K\"{o}rding, Ph.D.}
\smallskip
{\small{\color{skills}\textbf{Major Skills Obtained:}} Python-based data acquisition and analysis, Bayesian modeling of behavior, deep neural networks, deep reinforcement learning, pose-estimation (video-based MoCap)} \\
\smallskip
{\color{accent}\underline{\textbf{Projects:}}}\\
%\smallskip
\begin{adjustwidth}{0.5em}{0pt}
    \begin{itemize}
    % First project
    \item[\color{accent}\ding{224}] {\color{emphasis}\textbf{Bayesball: Bayesian Integration in Professional Baseball Batters (project lead) \hspace{10pt} \cvtag{\href{https://youtu.be/i4jBHfjPDnI}{ \faYoutubePlay}~~NMC 4.0 Presentation}}}\\
    \smallskip\vspace{-.75em}
        % Project highlights
        \begin{itemize}
            \item Used large-scale open-source baseball data (millions of pitches) to demonstrate that professional batters manage batting uncertainty in way that consistent with Bayesian statistics.
            \item First study to translate small-scale lab studies of Bayesian behavior to the real world
            \item {\color{pink} \textbf{Deliverable:}} Journal article (submitted), conference presentation, open-access python notebook
        \end{itemize}
    % \medskip    
    \smallskip
    % Second project
    \item[\color{accent}\ding{224}] {\color{emphasis}\textbf{Studying Movement in Naturalistic Environments Using Pose Estimation (project lead)}}\\
    \smallskip
        % Project highlights
        \begin{itemize}
            \item Used open source pose estimation techniques to track human movement and extend existing studies of motor control and Bayesian behavior in naturalistic environments
            \item {\color{pink} \textbf{Deliverable:}} Python notebooks ready for reuse in future projects 
        \end{itemize}
    \end{itemize}
\end{adjustwidth}
%\smallskip
\vspace{-0.5em}
\divider

\researchexperience{Graduate Research Assistant,(NIH F99/K00 Fellowship Recipient~\href{https://projectreporter.nih.gov/project_info_description.cfm?aid=9470585&icde=37206091}{\linkicon})}{Laboratory for Non-Invasive Brain Machine Interfaces \href{https://www.facebook.com/UHBMIST}{\linkicon}, Electrical \& Computer Engineering, University of Houston}{Aug 2014--Jan 2020}{Supervisor: Jose L. Contreras-Vidal, Ph.D.}
\smallskip
{\small{\color{skills}\textbf{Major Skills Obtained:}} Signal processing and machine learning for neural signals; data acquisition and analysis of EEG, EMG, IMU-based motion capture, goniometers, and fMRI; experience in the design and development of rehabilitation robotics, such as prosthetic limbs and exoskeletons; collaboration and communication in multi-disciplinary teams, including engineers, scientists, clinicians, and the end-user population} \\
\smallskip
{\color{accent}\underline{\textbf{Projects (Selected):}}}\\
% \medskip
\begin{adjustwidth}{0.5em}{0pt}
    \begin{itemize}
    % First project
    \item[\color{accent}\ding{224}] {\color{emphasis}\textbf{Brain-Machine Interfaces (BMIs) for Control of Prosthetic Devices (project lead) ~~\cvtag{\href{https://github.com/jabrantley/NEUROLEG}{\faGithub}~~Project Repo}}}\\
    \smallskip
        % Project highlights
        \begin{itemize}
            \item Utilized EEG, EMG, IMUs, and fMRI to investigate the cortical representation of phantom limb in amputees 
            \item Developed a closed-loop control framework (cleaning/prediction) using a nonlinear Kalman filter to predict phantom limb movements from EEG for control of an external robotic prosthesis. 
            \item Led a strong collaboration with a mechatronics research lab and an fMRI lab to complete the study. 
            \item {\color{pink} \textbf{Deliverable:}} Fully realized EEG-based real-time BMI, a published book chapter, two papers submitted/in preparation, mentored undergraduates
        \end{itemize}
    \medskip    
    
    % Second project
    \item[\color{accent}\ding{224}] {\color{emphasis}\textbf{Neural Correlates of Human Multi-Terrain Walking (project lead) \hspace{10pt} \cvtag{ \href{https://uh.edu/news-events/stories/2017/november/11302017Contreras-Vidal-Brain-Activity-Walking.php}{\faYoutubePlay}~,~\href{https://ieeexplore.ieee.org/document/8746863}{\faFileTextO}~~ Media highlights}}}\\
    \smallskip
        % Project highlights
        \begin{itemize}
            \item Developed a unique experimental framework for mobile brain and body imaging (MoBI) to record simultaneous EEG, EMG, and IMU-based motion capture during unconstrained walking.
            \item Developed offline decoding strategy using signal processing and machine learning for prediction of terrain transitions directly from brain signals.
            \item {\color{pink} \textbf{Deliverable:}} Complete experimental framework for MoBI data collection, published numerous journal and conference paper, published multi-modal data as open-access repository \href{https://figshare.com/articles/EEG_Data/5616109/5}{\faFloppyO}, mentored undergraduate and graduate students.
        \end{itemize}
    \medskip   
    
    % Third project
    \item[\color{accent}\ding{224}] {\color{emphasis}\textbf{Regulatory Concerns for Rehabilitation and Neurotechnology (project co-lead) \hspace{10pt}}}\\
    \smallskip
        % Project highlights
        \begin{itemize}
            \item {\color{pink} \textbf{Deliverable:}} Lead the section,"End-effectors: Actuators and Feedback" in the \textit{IEEE Neurotechnologies for Brain-Machine Interface Standards}; published two journal articles discussing regulatory and clinical concerns related to exoskeletons  and direct to consumer neurotechnology.  
        \end{itemize}    
    \end{itemize}
\end{adjustwidth}

%\smallskip
\vspace{-0.5em}
\divider

\researchexperience{Graduate Research Assistant}{Orthopaedic Biomechanics \& Biomaterials Laboratory\href{https://hsc.unm.edu/medicine/departments/ortho-rehab/orthopaedics/research/biomechanics.html}{\linkicon}, University of New Mexico}{Aug 2012--Aug 2014}{Supervisor: Mahmoud Reda Taha, Ph.D; Deana Mercer, MD; Christina Salas, Ph.D}
\smallskip
{\small{\color{skills}\textbf{Major Skills Obtained:}} Synthetic and cadaveric bone experimental testing, finite element modeling, collaboration and communication with surgeons}\\
\smallskip
{\color{accent}\underline{\textbf{Projects:}}}\\
% \medskip
\begin{adjustwidth}{0.5em}{0pt}
    \begin{itemize}
    % First project
    \item[\color{accent}\ding{224}] {\color{emphasis}\textbf{Experimental and computational investigation of Orthopaedic surgical techniques (project lead/ research assistant)}}\\
    \smallskip
        % Project highlights
        \begin{itemize}
            \item Utilized mechanical testing instruments and finite element modeling to design and validate treatment options for complex fractures and musculoskeltal conditions.
            \item Worked directly with orthopedic residents, fellows, and attending faculty members to develop novel engineering solutions to problems encountered in the operating room. 
            \item {\color{pink} \textbf{Deliverable:}} Experimental and computational results of surgical treatments, publications and conference presentations
        \end{itemize}
    % \medskip    
    \end{itemize}
\end{adjustwidth}

% \smallskip
% \divider%\clearpage

% \cveventDateRight{Building Research Achievement in Neuroscience (BRAiN)}{New Mexico State University\href{http://brain.nmsu.edu}{\linkicon}, University of Colorado Anschutz Medical Campus \href{http://www.ucdenver.edu/academics/colleges/medicalschool/programs/Neuroscience/Program/Pages/brain.aspx}{\linkicon}}{Aug 2011--May 2012}{}
% \begin{small} Advisors: Elba Serrano, Ph.D; Emily Gibson, Ph.D.; \\ and Diego Restrepo, Ph.D.  \end{small}
